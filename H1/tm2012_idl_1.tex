\documentclass[a4paper,12pt]{article}
%\documentclass[a4,landscape]{seminar}
%\slidesmag{1}
%\documentstyle[a4paper]{letter}
%\documentstyle[fullpage]{letter}
\usepackage{graphicx} 
\usepackage{amsmath} 
\usepackage{color}

%\usepackage{pstcol} % To use the standard `color' package with Seminar
%\usepackage{semcolor}
%\usepackage{slidesec}
%\input{seminar.bug}
%\input{seminar.bg2} % See the Seminar bugs list
%\usepackage{pst-grad}

\definecolor{LemonChiffon}{rgb}{1.,0.98,0.8}
\definecolor{Gold}{rgb}{1.,0.84,0.}
\definecolor{darkblue}{rgb}{0.25,0.,0.95}
\definecolor{lightblue}{rgb}{0.4,0.65,0.95}
%\slideframe[\psset{fillstyle=gradient,gradmidpoint=0.0,
%                   gradbegin=lightblue,gradend=darkblue}]{scplain}

\def\pp{\mathaccent "7F }
\def\p{\mathaccent 95 }


\def\sini {\sin i}
\def\cosi {\cos i}

\def\sino {\sin \Omega}
\def\coso {\cos \Omega}

\def\sinw {\sin \omega}
\def\cosw {\cos \omega}



\begin{document}
\font\norm=cmr12
\font\hugeb=cmb22
\font\isob=cmb18
\font\iso=cmr18
\font\med=cmr14
\font\medb=cmb14
\baselineskip 0.6cm
\parskip\medskipamount
\parskip 0.5cm
\parindent 5mm

%\boldmath

\norm
\hsize 16cm
\vsize 30cm
\newcommand{\bul}{$\bullet \ \ $}
\newcommand{\buu}{\hskip 0.5cm}
\newcommand{\arrow}{$\Rightarrow \ \ $}
\newcommand{\barrow}{\hskip 0.5cm $\Rightarrow \ \ $}
\newcommand{\larrow}{\hskip 0.5cm $\Leftarrow \ \ $}
\newcommand{\hlin}{\vskip 0.5cm}
\newcommand{\hpar}{\vskip 1.0cm}
\newcommand{\nin}{\noindent}
\def\pii{\tilde{\omega}_\circ}

\newcommand{\VV}{$\vec V$ \hskip 0.1cm}
\newcommand{\RR}{$\vec R$ \hskip 0.1cm}

\newcommand{\VVE}{\vec V}
\newcommand{\RRE}{\vec R}

%\pagestyle{empty}
{\centerline{}




{\norm
\vskip 0cm
{{\centerline { {\isob CELESTIAL MECHANICS (Fall 2012): }}}}
\vskip 0.2cm
{{\centerline { {\isob COMPUTER EXERCISES I}}}}

{{\centerline { { (Heikki Salo 5.10.2010)}}}}

\vskip 2cm
{\medb 1. Solution of Kepler's equation}

{\medb 2. Study elliptical orbits}

{\medb 3. Compare different anomalies: $E$, $f$, $M$}

{\medb 4. Numerical evaluation of time averages on elliptical orbit}



\newpage

{\isob 1. Solution of Kepler's equation} (kepler.pro, kepler\_demo.pro)


The position \RR and velocity \VV in an elliptic orbit 
can not be expressed in a closed form as a function of time $t$.  However,
\RR and \VV can calculated either by using the true anomaly
$f$ or the eccentric anomaly $E$. To obtain either $f$ or $E$ one
needs to first solve the Kepler's equation, connecting $E$ and the
mean anomaly $M$:
\begin{equation}
M = \sqrt{\mu/a^3} (t-\tau) = E - \epsilon \sin E
\end{equation}
\noindent where
\begin{eqnarray}
\mu  &=& G (m_1 +m_2), \ G= {\rm gravitational \ constant}, \ m_1, m_2 = {\rm masses} \nonumber \\
\tau &=& { \rm pericenter \ time}, \ a   = { \rm semimajor \ axis}, \ \epsilon =  { \rm eccentricity} \nonumber
\end{eqnarray}
\noindent
To solve $E$ as a function of $M$ from Kepler's equation, $M = E - \epsilon \sin E$, one can use for example substitution iteration. Denote
$x= \epsilon \sin E $, so that $M=E-x$. From this $E$ can be expressed in the form $E=M+x$, and one can set up an iteration
\begin{equation}
E_{i+1} = M + \epsilon \sin E_i, \ \  i=0,1,2,...
\label{kepler_subs}
\end{equation}
\noindent using for example $E_0=M$ as the first approximation. The
iteration is continued until $|E_{i+1} -E_i|$ is sufficiently small,
say less than $10^{-10}$.

Another possible way to solve Kepler's equation is to use 
Newton's method for finding the root of the equation $g(E) = E -\epsilon \sin E -M = 0$.  This is done by the iteration schema
\begin{equation}
E_{i+1} = E_i - g(E_i)/g'(E_i), \ \  i=0,1,2,...
\end{equation}
\noindent where $g' (E_i)= dg(E_i)/dE_i = 1 - \epsilon \cos E_i$.
Using here $E_0=M$ as a first guess gives $E_1=M +\epsilon \sin M/(1-\epsilon \cos M )$, which can also be used as an improved starting value in the schema of Eq. \ref{kepler_subs}.

{\bf WHAT TO DO:}  a) Write an IDL-procedure that solves $E$ for a given $M$
and $\epsilon$, using either of the above described methods.  b) Solve
the eccentric anomaly $E$ corresponding to mean anomaly $M= 45^\circ$,
for $\epsilon =0.01, \ 0.05, \ 0.50, \ 0.90, \ 0.99$. How many iteration steps
are required to reach an accuracy of $0.01^\circ$ in each case?
Try also what happens for $M=359^\circ$! 



\newpage

{\isob 2. Study elliptical orbits} \ (elliptic\_demo.pro)


a) In terms of eccentric anomaly $E$ the position in elliptic orbit
can be represented as
\begin{equation}
\vec R = \vec A (\cos E - \epsilon) + \vec B \sin E
\label{eq_vec_r}
\end{equation}
\noindent where
$\vec A $ and $\vec B$ are perpendicular vectors in the plane of the orbit, and $\vec A$ points to the pericenter of the orbit ($|\vec A| = a, \  |\vec B| = b =a \sqrt{1-\epsilon^2}) $. Choose the x-axis parallel to $\vec A$ and plot
elliptic orbits with eccentricity $\epsilon=0.0, 0.1, ..., 0.8$.
Use $a=1$ for the semimajor axis, and use for example 200 uniformly chosen
$E$ values from $0$ to $2 \pi$ to calculate $(x,y)$ points defining the orbit.


b) In terms of the true anomaly $f$ (= polar angle from the pericenter), the length of the radius vector is given by
\begin{equation}
r = |\vec R| = \frac{a(1-\epsilon^2)}{1+\epsilon \cos f}
\end{equation}
\noindent As above, take 200 $f$ values from $0$ to $2 \pi$, and
calculate $(x,y)$ points defining the orbit.

c) Uniformly chosen $E$ or $f$ values, while easy to use for plotting the
orbit, do not represent the movement of the particle in its orbit
during equal time intervals. To do this, one has to take equally
spaced points in time, or what is equivalent, equally spaced values
of mean anomaly. Repeat what was done in a) and b) by choosing
200 uniformly chosen $M$ values from $0$ to $2 \pi$. For each of these,
solve Kepler's equation to get $E$.

d) Make also a plot with fewer points (say 50), for $\epsilon =0.0$ and $0.8$,
illustrating the change in speed at various parts of the orbit.


e) Make also plots of the length of the radius vector as a function of time.
Choose equally spaced $M$ values, calculate $E$, and use the equation
\begin{equation}
r = |\vec R| = a(1 - \epsilon \cos E)
\end{equation}

Also check how the central force varies with time (proportional to
$1/r^2$).  Plot also $1/r^3$ versus time: $2D/r^3$ represents the
magnitude of the tidal force a particle with a diameter $D$ would
experience (tidal force = difference in the force felt by the
different parts of the particle -- differentiate $1/r^2$ with respect to $r$)

f) Play also with velocity. In terms of $E$,
\begin{equation}
\vec V =\p {\vec R} = (- \vec A \sin E + \vec B \cos E) \p E,
\label{eq_vec_v}
\end{equation}
\noindent where the derivative $\p E$ is obtained by differentiating Kepler's
equation,
\begin{equation}
\p E = \frac{\sqrt{\mu/a^3}}{(1-\epsilon \cos E)}
\label{eq_ep}
\end{equation}
\noindent You can choose the units by setting $\mu=1,\  a=1$
(what is the orbital period with these values?)

From the velocity components one can evaluate numerically
the speed $v= |\vec V|$ as a function of time (via $E$), as well as
the  radial velocity $v_r= \vec V \cdot \vec R/r$, and the
tangential velocity $v_t=|\vec V - v_r \vec R/r|$.
Compare these with the analytical formulas derived in the lectures (exercise I.6),
\begin{eqnarray}
v_r &=& v_{circ} \ a/r \ \epsilon \ \sin E \\ \nonumber
v_t &=& v_{circ} \ b/r \\ \nonumber
v   &=& v_{circ} \ \sqrt{\frac{1+\epsilon \cos E}{1-\epsilon \cos E}}
\label{eq_vr}
\end{eqnarray}
\noindent where $v_{circ}= \sqrt{\mu/a}$ is the speed on a circular orbit.

g) Finally, study how the kinetic energy and potential energy (of the
relative orbit per unit mass) vary on different positions in the eccentric orbit, $E_{kin}=\frac{1}{2}v^2$ and $E_{pot}=-\mu/r$. Also verify that their sum equals the total energy
$h=-\frac{\mu}{2a}$.
Also check that the angular momentum $\vec R \times \vec V$ is constant, with
absolute value of $\sqrt{\mu a(1-\epsilon^2)}$.
}

\newpage

{\isob 3. Compare different anomalies: $E$, $f$, $M$} \ (ano\_demo.pro)

Using the previous results, you can choose equally spaced $M$ values,
and calculate the corresponding $E$. From $E$ you can obtain $f$,
with the formulas derived in lectures
\begin{eqnarray}
\cos f &=& \frac{\cos E - \epsilon}{1-\epsilon \cos E}, \\ \\ \nonumber
\sin f &=& \frac{\sqrt{1-\epsilon^2} \sin E}{1-\epsilon \cos E},
\end{eqnarray}
\noindent and using the IDL arctan-function with two arguments,
$f=${\bf atan(sin f, cos f)}.
Make plots showing $E$ and $f$ vs. $M$ for different eccentricities.

Plot also $E-M$ and $f-M$ vs. $M$, and compare the differences to
the first terms of series expansions
\begin{eqnarray}
E &=& M + \epsilon \sin M + \epsilon^2 (\frac{1}{2} \sin 2M) + \epsilon^3 (\frac{3}{8} \sin 3M - \frac{1}{8} \sin M) + ..., \\ \nonumber
f &=& M + 2 \epsilon \sin M + \epsilon^2 (\frac{5}{4} \sin 2M) + \epsilon^3 (\frac{13}{12} \sin 3M - \frac{1}{4} \sin M) + ...
\end{eqnarray}

\newpage

{\isob 4. Numerical evaluation of time averages} \ (aver\_demo.pro)

In the lectures (exercise I.7) the time averages of various functions
(e.g. $<r>,\ <v^2>)$) were evaluated analytically, by utilizing a change
of variables from time to either eccentric anomaly or to true anomaly.
Make the same numerically, utilizing the fact that
\begin{equation}
<x> = \frac{1}{2\pi} \int_0^{2\pi} x(m) \ dM
\end{equation}
\noindent which can be numerically evaluated as
\begin{equation}
<x> = \frac{1}{n} \sum x(M_i), \ \  M_i = i \ (2\pi/n),\ \ i=0,..., n-1
\end{equation}
\noindent Do this for quantities $r,r^2,1/r,1/r^{-2},1/r^{-3},v^2$,
for different eccentricities, and compare to analytic formulas derived in exercise I.7:



%\begin{equation}
%<r> = a(1+\epsilon^2/2),  \ \  <r^2> =a^2(1+3 \epsilon^2/2),  \ \ <1/r> = 1/a
%\end{equation}
%\begin{equation}
%<1/r^2> = 1/a^2 (1-\epsilon^2)^{-1/2}, \ \ <1/r^3> = 1/a^3 (1-\epsilon^2)^{-3/2}, \ \ <v^2> = \mu/a
%\end{equation}


\begin{eqnarray}
<r> &=& a(1+\epsilon^2/2),  \nonumber \cr
<r^2> &=& a^2(1+3 \epsilon^2/2), \nonumber \cr
<1/r> &=& 1/a, \nonumber \cr \cr
<1/r^2> &=& 1/a^2 (1-\epsilon^2)^{-1/2},\nonumber \cr
<1/r^3> &=& 1/a^3 (1-\epsilon^2)^{-3/2},\nonumber \cr \cr
<v^2> &=& \mu/a\nonumber
\end{eqnarray}



%-------------------------------------------------------------------------



\end{document}









