\documentclass[a4paper,12pt]{article}
%\usepackage[pdftex]{graphicx} 
%\usepackage{indentfirst}
%\usepackage[varg]{txfonts}
%\usepackage{times} 
%\usepackage{colortbl}
%\usepackage{amsmath}
%\usepackage{amssymb}

%\usepackage[T1]{fontenc}
%\usepackage[finnish]{babel}


%\usepackage[pdftex]{geometry}  %resizing headers
%\usepackage[pdftex]{thumbpdf}  %create thumbnails
\usepackage[pdftex]{graphicx}
\usepackage{times}             %fonts
%\usepackage[pdftex,bookmarks]{hyperref}
%\usepackage[verbose]{layout}
\usepackage{ifthen}
\usepackage{colortbl}
\usepackage{amsmath}
\usepackage{amssymb}
\usepackage[T1]{fontenc}
\usepackage[finnish]{babel}



\topmargin=-0.9cm
\oddsidemargin=0.0cm
\evensidemargin=0.0cm
\addtolength{\textheight}{0.5cm}


\definecolor{LemonChiffon}{rgb}{1.,0.98,0.8}
\definecolor{Gold}{rgb}{1.,0.84,0.}
\definecolor{darkblue}{rgb}{0.25,0.,0.95}
\definecolor{lightblue}{rgb}{0.4,0.65,0.95}
\definecolor{grayred}{rgb}{0.7,0.,0.} 
\definecolor{red2}{rgb}{1.0,0.5,0.} 
\definecolor{LemonChiffon}{rgb}{1.,0.98,0.8}
\definecolor{Gold}{rgb}{1.,0.84,0.}
\definecolor{darkblue}{rgb}{0.25,0.,0.95}
\definecolor{lightblue}{rgb}{0.4,0.65,0.95}
\definecolor{RGBblack}{rgb}{0.0,0.0,0.0} 
\def\red{\color{red}}
\def\green{\color{green}}
\def\white{\color{white}}
\def\white{\color{LemonChiffon}}
%\def\blue{\color{lightblue}}
\def\blue{\color{RGBblack}}
\def\gold{\color{Gold}}
\def\black{\color{RGBblack}}



\def\pp{\mathaccent "7F }
\def\p{\mathaccent 95 }



\def\sini {\sin i}
\def\cosi {\cos i}

\def\sino {\sin \Omega}
\def\coso {\cos \Omega}

\def\sinw {\sin \omega}
\def\cosw {\cos \omega}



\begin{document}
\font\norm=cmr12
\font\hugeb=cmb22
\font\isob=cmb18
\font\iso=cmr18
\font\med=cmr14
\font\medb=cmb14
\baselineskip 0.6cm
\parskip\medskipamount
\parskip 0.05cm
\parindent 5mm

%\boldmath

\norm
\hsize 16cm
\vsize 30cm

\def\pii{\tilde{\omega}_\circ}

\newcommand{\VV}{$\vec V$ \hskip 0.1cm}
\newcommand{\RR}{$\vec R$ \hskip 0.1cm}

\newcommand{\VVE}{\vec V}
\newcommand{\RRE}{\vec R}


\newcommand{\tsub}[2]{#1_{\mbox{\tiny #2}}}
\newcommand{\psig}{\frac{\partial p}{\partial \sigma}}
\newcommand{\dchi}{\frac{\partial }{\partial \chi}}
\newcommand{\dtheta}{\frac{\partial }{\partial \theta}}
\newcommand{\ddchi}{\frac{\partial^2 }{\partial \chi^2}}

\newcommand{\bul}{$\bullet \ \ $}
\newcommand{\buu}{\hskip 0.5cm}
\newcommand{\buv}{\hskip 1.25cm}
\newcommand{\arrow}{$\Rightarrow \ \ $}
\newcommand{\barrow}{\hskip 0.5cm $\Rightarrow \ \ $}
\newcommand{\larrow}{\hskip 0.5cm $\Leftarrow \ \ $}
\newcommand{\hlin}{\vskip 0.2cm}
\newcommand{\hpar}{\vskip 0.4cm}
\newcommand{\nin}{\noindent}
\def\pii{\tilde{\omega}_\circ}
\def\ppder{\mathaccent "7F }
\def\pder{\mathaccent 95 }
\def\sini {\sin i}
\def\cosi {\cos i}

\def\sino {\sin \Omega}
\def\coso {\cos \Omega}

\def\sinw {\sin \omega}
\def\cosw {\cos \omega}


\newcommand{\sbul}{\scriptsize $\bullet$} 

\newcommand{\beq}{\begin{eqnarray}}
\newcommand{\eeq}{\end{eqnarray}}
\newcommand{\prob}{{\rm prob}}
\newcommand{\pd}[2]{\frac{\partial #1}{\partial #2}}
\newcommand{\pdpd}[2]{\frac{\partial^2 #1}{{\partial #2}^2}}
\newcommand{\ppd}[3]{\frac{\partial #1}{\partial #2 \partial #3}}

\def\pii{\tilde{\omega}_\circ}
%\def\�{\"a}
%\def\�{\"A}
%\def\�{\"o}
%\def\�{\"O}

\def\sx{{\sigma_x}}
\def\sy{{\sigma_y}}
\def\sx2{{\sigma_x}^2}
\def\sy2{{\sigma_y}^2}

\def\me{$M_{Earth}$}
\def\msun{$M_{Sun}$}
\boldmath


\def\kvec{{\vec K}}
\def\evec{{\vec e}}
\def\rvec{{\vec R}}
\def\vvec{{\vec V}}
\def\ovec{{\vec \omega}}
\def\rpvec{\pder {\vec R}}
\def\rppvec{\ppder {\vec R}}
\def\rtvec{{{\vec R}^\prime}}

\def\dt{\Delta t}
\def\ddt{{(\Delta t)}^2}
\def\dddt{{(\Delta t)}^3}
\def\half{\frac{1}{2}}
\def\sixth{\frac{1}{6}}




%\pagestyle{empty}
{\centerline{}




{\norm
\vskip 0cm
{{\centerline { {\isob CELESTIAL MECHANICS (Fall 2012): }}}}
\vskip 0.2cm
{{\centerline { {\isob COMPUTER EXERCISES III}}}}

{{\centerline { { (Heikki Salo 26.11.2012)}}}}}

\vskip 2cm

{\medb 8. Numerical integration of 2-body relative orbit}

\buv Using I and II order Taylor series

\buv Using Runge-Kutta 4 method


\vskip 1cm

%Example routines can be copied from

%$\sim$hsalo/TM2006$\_$DEMO3.dir



\newpage

{\medb 8. Numerical integration of 2-body relative orbit}

\hpar

\bul Dynamical equation for two-body relative motion,

\hskip 6cm {\blue 
$$ \ppder {\vec R} = -G(m_1+m_2) \frac{\vec R}{r^3} = -\mu \frac{\vec R}{r^3}$$
}


\bul General principle: divide the second order differential equation for radius vector $\rvec$ 

\buu into two first order differential equations for vectors $\rvec$ and $\vvec${\blue  
\beq
\pder {\vec R} &=& \vec V \nonumber \\
\pder {\vec V} &=& \ppder {\vec R} \nonumber
\eeq
}

\bul Choose a Cartesian coordinate system $\rvec = [x,y,z]$, $\vvec=[v_x,v_y,v_z]$

\buu In component form:  \hskip 2cm {\blue ($r=\sqrt{x^2+y^2+x^2}$)})
{\blue
\beq
\pder x &=& v_x  \nonumber \cr
\pder y &=& v_y  \nonumber \\
\pder z &=& v_z  \nonumber \\
{\pder v}_x &=& \ppder x \hskip 1cm = ~~ -\mu~ x/r^3  \nonumber \cr
{\pder v}_y &=& \ppder y \hskip 1cm = ~~ -\mu~ y/r^3 \nonumber \cr
{\pder v}_z &=& \ppder z \hskip 1cm = ~~ -\mu~ z/r^3 \nonumber
\eeq}

\bul This is an {\em initial value problem (alkuarvoprobleema) }: position, velocity known 

\buu at a given instant of time  \barrow orbit for any desired time interval.

\buu 6 initial values needed

\hpar
\hpar

\bul We next look at some simple methods for the solution of this problem:

\buu truncated Taylor expansions, Leap-frog method, RK4

\newpage



{\bf \red \large A. ~ First order truncated Taylor expansion:} 

\buu \buu {\blue $f(t+\Delta t)= f(t) +f'(t) \Delta t$}  


\hpar
\hpar
\bul Simplest possible method of solution for any first order differential equation
\hlin 

\buu 
i) at time {\blue $t=t_o$} the position {\blue $[x_0, y_0, z_0]$} and velocity {\blue $[(v_x)_0, (v_y)_0, (v_z)_0]$}
\hlin 
\buu ii) Approximation at time {\blue $t_1=t_0+\Delta t$: }
{\blue
\beq 
x_1 &=& x_0 + \pder x~ \dt  \hskip 3cm  ~~~~~~\pder x =  (v_x)_0 \nonumber \cr
y_1 &=& y_0 + \pder y~ \dt  \hskip 3cm  ~~~~~~\pder y =  (v_y)_0 \nonumber \cr
z_1 &=& z_0 + \pder z~ \dt  \hskip 3cm  ~~~~~~\pder z =  (v_z)_0 \nonumber \cr
(v_x)_1 &=& (v_x)_0 + \pder {v_x}~ \dt \hskip 3.3cm  \pder {v_x} = \ppder x_0= ~-\mu  ~x_0/{r_0}^3 \nonumber \cr
(v_y)_1 &=& (v_y)_0 + \pder {v_y}~ \dt \hskip 3.3cm  \pder {v_y} = \ppder y_0= ~-\mu  ~y_0/{r_0}^3  \nonumber \cr
(v_z)_1 &=& (v_z)_0 + \pder {v_z}~ \dt \hskip 3.3cm  \pder {v_z} = \ppder z_0= ~-\mu ~ z_0/{r_0}^3  \nonumber
\eeq }
\buv \buv  {\small (NOTE: derivatives evaluated at $t_0$ \barrow must use $r_0=\sqrt{{x_0}^2+{y_0}^2+{z_0}^2}$) 

\hskip 11cm in velocity derivatives}
\hlin

\buu iii) Repeat the step ii) for the next time steps {\blue $t_i \rightarrow t_i+\Delta t$},

\buv until the whole desired time interval is covered

\hpar \bul Easy to program, extremely inaccurate

\buv error/step proportional to $\ddt$

\buv at the time interval $t_{tot}$ one needs $t_{tot}/\dt$ steps

\buu \barrow (in the worst case) the total error proportional to $\dt$

\hlin

\buu {\red \barrow NEVER USED!}  {\scriptsize (except in initial tests)}




\newpage


{\bf \large \red B.~ Second order Taylor expansion:} 

\buu \buu {\blue $f(t+\Delta t)= f(t) +f'(t) \dt + \half f''(t) \ddt$}
\hpar

\hpar
\buu Approximation at time {\blue $~t_1=t_0+\Delta t$: }
{\blue
\beq 
x_1 &=& x_0 + \pder x~ \dt +\half \ppder x~ \ddt \nonumber \cr 
y_1 &=& y_0 + \pder y~ \dt +\half \ppder y~ \ddt  \nonumber \cr 
z_1 &=& z_0 + \pder z~ \dt +\half \ppder z~ \ddt  \nonumber \cr 
(v_x)_1 &=& (v_x)_0 + \pder {v_x}~ \dt +\half {\ppder v}_x~ \ddt \nonumber \cr
(v_y)_1 &=& (v_y)_0 + \pder {v_y}~ \dt +\half {\ppder v}_y~ \ddt \nonumber \cr
(v_z)_1 &=& (v_z)_0 + \pder {v_z}~ \dt +\half {\ppder v}_z~ \ddt \nonumber
\eeq
}
\buu Here {\blue ${\ppder v}_x = \dddot{x}$}, {\blue${\ppder v}_y = \dddot{y}$}
and {\blue ${\ppder v}_z = \dddot{z}$}.

\hlin
\bul How to construct $\dddot{\vec R}_0 \equiv [\dddot{x_0},\dddot{y_0}, \dddot{z_0}]$?
\hlin

\buu Obtain $\ddot{\vec R}$ by differentiating $-\mu {\vec R}/r^3$ \barrow
{\blue
$$\dddot{\vec R} = d/dt~(\ddot{\vec R})= -\mu \frac{\dot{\vec R}r^3 -3r^2 \dot{r} \vec{R}}{r^6} = -\mu/r^3 \left(\dot{\vec R} - 3 \frac{(\rvec \cdot\rpvec) \rvec}{r^2}\right),$$,

\buu {\small  $\pder r$ is calculated using the formula $r^2= \rvec \cdot \rvec$ \barrow $d(r^2)/dt = 2 r \pder r \equiv 2 \rvec \cdot \rpvec$ }
\hpar
}
\bul In the component form
{\blue \small
\hlin

\buv \buu  $\dddot{x}=  -\mu/r^3 \left( v_x - 3 \frac{(\rvec \cdot\rpvec)}{r^2} x\right)$
\hskip 4cm $\rvec  \cdot \rpvec = x v_x + y v_y  + z v_z$

\buv \buu $\dddot{y}=  -\mu/r^3 \left( v_y - 3 \frac{(\rvec \cdot\rpvec)}{r^2} y\right)$
\hskip 4cm $r = \sqrt{x^2 + y^2  + z^2}$

\buv \buu $\dddot{z}=  -\mu/r^3 \left( v_z - 3 \frac{(\rvec \cdot\rpvec)}{r^2} z\right)$
}

\hlin

\bul The error/step is proportional to $(\dt)^3$ 

\buu (worst case total
error proportional to $\ddt$)

 \barrow allows longer steps than the first order Taylor expansion, still not very good


\newpage


{\large \red \bf C.~  Higher order Taylor expansions:} 

\buu \buu {\blue $f(t+\dt)= f(t) + \sum \frac{1}{k!} f^{(k)}(t) (\Delta t)^k$}

\hlin

\hpar
\bul In practice the II order method is far too inaccurate

\buu \barrow  higher order methods (e.g. 8th order etc.)
\hlin

\bul Calculation of higher order derivatives of $\ddot{\vec R}$ gets rapidly
very cumbersome

\barrow tabulated $f$ and $g$ series:  
\hpar
\buv {\blue $\rvec (\tau) = f(\tau, \rvec_0,  \rtvec_0) ~ \rvec_o + g(\tau, \rvec_0, \rtvec_0) ~ \rtvec_0$}
\hlin
\buv {\blue $\vvec (\tau) = \pder f(\tau, \rvec_0,  \rtvec_0) ~ \rvec_o + \pder g(\tau, \rvec_0, \rtvec_0) ~ \rtvec_0$}
\hpar
{\small
\buu where

\buv $\tau \equiv \sqrt{\mu} (t-t_0)$  and

\buv $^\prime$ denotes differentiation with respect to normalized time

\buv \buv ($d/dt = (d/d\tau) (d\tau/dt) = \sqrt{\mu} ~d/d\tau$)

\buv $f= 1- \half {\tau}^2 {r_0}^3 +\half {\tau}^3 {r_0}^{-4}{r_0}^\prime ...$

\buv $g= \tau  - \frac{1}{6} \tau^3 {r_0}^{-3} ...$

\buv  (here truncated with the accuracy that corresponds to II order Taylor
expansion}


\hpar

\newpage
{\red \bf D 'TIME-CENTERED LEAPFROG'}

\hlin
\bul A method sometimes used in N-body simulations: the accuracy corresponds to second 

\buu degree Taylor expansion,  but the method does not require the calculation of $\dddot
{\rvec} = \dot{\vec F}$

\hlin

\bul The positions and velocities are approximated with formulas which are

\buu symmetric with respect to time ({\em time-centered})


\buu For example, the $x$ component ($y$ and $z$ are treated in a similar manner)
{\blue
\beq
v_x(t+\dt/2) &=&v_x(t-\dt/2) + f_x(t) \dt \nonumber \\
x(t+\dt)&=&x(t)+ v_x(t+\dt/2) \dt  \nonumber
\eeq
}
%

{\buv   \small It is easy to see that the accuracy corresponds to II degree Taylor:

\buv  write the Taylor expansions
\beq
v_x(t+\dt/2) = v_x(t) + f_x(t) \dt/2  + \frac{1}{2} \pder f_x(t) (\dt/2)^2 \nonumber \\
v_x(t-\dt/2) = v_x(t) - f_x(t) \dt/2  + \frac{1}{2} \pder f_x(t) (\dt/2)^2 \nonumber
\eeq
%
\buv \buv subtract the latter from the first \barrow
%
\beq
v_x(t+\dt/2) - v_x(t-\dt/2) =  f_x(t) \dt ~ + ~ ... \dt^3 \nonumber
\eeq
%

\buv leads to leapfrog formula, with an error term proportional to $\dt^3$

\buv Same accuracy holds for the position vector components}
\hlin


\newpage
\large {\red \bf E. ~ Fourth order Runge-Kutta method (RK4)}
\hlin
\bul A method accurate to $(\dt)^4$, without the need to calculate explicitly the 

\buu the higher order derivatives of force.

\bul The accuracy is achieved by evaluating the forces in 4 different
intermediate steps.
\hlin
\bul Assume that\buu  $dx/dt=f(t,x)$, then
\hlin
{\blue
\buv $x(t+\dt) = x(t)+ \frac{\dt}{6} (k_1+2k_2+2k_3+k_4)$ where
\hlin

\buv \buv $k_1= f(t, x(t))$

\buv \buv $k_2= f(t +\frac{\dt}{2}, x(t)+\frac{\dt}{2}k_1)$

\buv \buv$k_3= f(t +\frac{\dt}{2}, x(t)+\frac{\dt}{2}k_2)$

\buv \buv$k_4= f(t +\dt, x(t)+\dt k_3)$
}
\hpar

\buu In practice most programming packages have subroutines
for these steps

\bul IDL contains a library-routine RK4: to use it one needs to 

\buu - collect the components of $\rvec$ and $\vvec$ into a single variable

\buv   to be integrated, say, $yy=[x,y,z,vx,vy,vz]$

\buu - make a loop over time steps, calling RK4 at each step for updating the variable $yy$

\buu - provide a function routine for returning the derivatives of $yy$

\hpar

{\small
\baselineskip 0.19cm
-------------------------------------------

\buv variable yy=[x,y,z,vx,vy,vz] integrated over the step dt

\buv func(time,yy) returns the derivate of yy  (=[dx/dt,dy/dt,dz/dt,d(vx)/dt,d(vy)/dt, d(vz)/dt])

\hlin

-------------------------------------------

 \buu time=0.

 \buu for i=0,n do begin

 \buv  dydx=func(time,yy)

 \buv   res=rk4(yy,dydx,time,dt,'func',/double)

 \buv   time=time+dt

 \buv   yy=res

 \buu endfor

-------------------------------------------

\hpar

\buu function func,t,yy

\buv x=yy(0) \& y=yy(1) \& z=zz(2) \& vx=yy(3) \& vy=yy(4) \& vz=yy(5) 

\buv ...

\buv dydx=yy*0.

\buv dydx(0)= ...

\buu return,dydx

\buu end
}
\newpage


{\red WHAT TO DO?} \hskip 1cm 


\bul Make a program to numerically integrate the 2-body relative motion.
Also compare to analytical solution, and check the conservation of
energy and angular momentum.


\hlin
\bul Analytical solution: Choose values for the orbital elements: $a, \epsilon, t_0 $. Also, set the units by choosing $\mu=G(m_1+m_2) = 1$. The orbital period
$P= 2\pi \sqrt{a^3/\mu}$ ~(therefore if $\mu=1$ \barrow  $P=2\pi$ at the unit distance)

\buu Solve Kepler's equation $t$:

{\blue
\buv $M= 2\pi \frac{t-t_0}{P}$

\buv $M= E- \epsilon \sin E$
}

\buu Position and velocity ($b=a \sqrt{1-\epsilon^2}$)

{\blue
\buv $x= a (\cos E - \epsilon)$

\buv $y= b \sin E$

\buv $z= 0$

\buv $v_x= -a \sin E ~ \sqrt{\mu/a^3} ~(1-\epsilon \cos E)^{-1}$

\buv $v_y= ~~ b \cos E ~ \sqrt{\mu/a^3} ~(1-\epsilon \cos E)^{-1}$

\buv $v_z=  0$
}

\buu At $t=t_0$ \barrow $M=0, E=0$ \barrow $[x_0,y_0,0]$, $[{v_x}_0,{v_y}_0,0]$

\hpar

\bul Numerical solution: 

\buu Integrate the orbit numerically for times
$t=t_0 + i ~~ \dt $, $~~i=0,1,2,...$. 

\buu Use the analytical solution for the
calculation of initial values.

\buu Compare to analytical solution evaluated for the same times

\buu Check the energy and angular momentum



\buv $h= \half v^2 -\mu/r$

\buv $|\vec k|= (\rvec \times \vvec)_z = x v_y-y v_x$




\bul Compare I and II order Taylor expansions, RK4, 

\buu using different time steps
(for example $\dt=0.1, 0.01, 0.001$ orbital periods).





\end{document}









