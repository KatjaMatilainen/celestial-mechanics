\documentclass[a4paper,12pt]{article}
\usepackage{natbib}
\usepackage{graphicx}
\usepackage{amssymb}
\usepackage{indentfirst}
\usepackage{amsmath} 
\usepackage[varg]{txfonts}
\usepackage{colortbl}
\topmargin=-0.75cm
\oddsidemargin=0.0cm
\evensidemargin=0.0cm

\definecolor{PastellGreen}{rgb}{0.5,0.8,0.6}
\definecolor{DarkGreen}{rgb}{0.0,0.9,0.0}
\definecolor{RGBblack}{rgb}{0.0,0.0,0.0} 
\definecolor{gray}{rgb}{0.3,0.3,0.3} 
\definecolor{lightgray}{rgb}{0.9,0.9,0.9} 
\definecolor{grayred}{rgb}{0.7,0.,0.} 
\definecolor{red2}{rgb}{1.0,0.5,0.} 
\definecolor{LemonChiffon}{rgb}{1.,0.98,0.8}
\definecolor{Gold}{rgb}{1.,0.84,0.}
\definecolor{darkblue}{rgb}{0.25,0.,0.95}
\definecolor{lightblue}{rgb}{0.4,0.65,0.95}
\def\red{\color{red}}
\def\green{\color{green}}
\def\white{\color{white}}
\def\white{\color{LemonChiffon}}
\def\blue{\color{lightblue}}
\def\gold{\color{Gold}}
\def\black{\color{RGBblack}}


\newcommand{\sverb}{\begin{verbatim}}
\newcommand{\everb}{\end{verbatim}}

\baselineskip 0.0cm
\begin{document}
\scriptsize



\small
\begin{verbatim}

--------------------------
EXAMPLES OF USING RK_INTE1
--------------------------

- rk_inte1.pro:
  rk4-integration (or Taylor-series) of 2-body orbit
  for given orbital elements and time interval 
  checks the conservation of orbital elements
                -"-          angular momentum L
                -"-          energy E
  plots the orbit + analytial solution
- prints the following instruction when called without parameters:
  IDL> rk_inte1

\end{verbatim}
\scriptsize
\begin{verbatim}
-----------------------------------------------------------------------------
pro rk_inte1,elem,t1,t2,dt,output=output,plot=plot
-----------------------------------------------------------------------------
Cartesian integration of non-perturbed 2-body orbit HS 20.11.02/12.02.06
  elem=[a,e,i,ome,w,tau]  initial orbital elements
  t1,t2                   integration time interval (in orbital periods)
  dt                      time step (in orbital periods)
KEYWORDS:
-----------------------------------------------------------------------------
  myy                     G * (m1+m2)   def=1.
  taylor=choice           use Taylor series, with degree=taylor (def=rk4)
                          choices=  1, 2               explicitly written                   
                                   -1,-2,-3,-4,-5,-6   using f,g-series
-----------------------------------------------------------------------------
PLOTTING KEYWORDS:
  plot=istep              plot every istep steps (def=no plot)
  oplot=color             plot on top of previous orbit with
                          color=oplot+2 (i.e 1->col=3=green)
  /connect                -> connect orbit points in the plot (def=no)
  wid                     limit of the plot region (DEF=1.25 a)
  /cplot                  plot analytic solution (white squares)
  title                   -> plot title
  output=val              output interval of ELEM,L,E in steps (def=nsteps/10)
                          val=negative -> just store
-----------------------------------------------------------------------------
OUTPUT/STORE KEYWORDS:
  t_out,l_out,e_out       dL/L and dE/E vs t_out (stored every |output| step)
  x_out,y_out,z_out       positions
  vx_out,vy_out,vz_out    velocities
  dl,de                   return averaged change in dL/L and dE/E
                                                    /orbit period
  /silent                 -> do not print anything to terminal
-----------------------------------------------------------------------------
EXAMPLE INPUT VALUES:
  /example                example of integration:
                          a=1,ecc=0.5,i=10,ome=90.,w=0,tau=0
                                 t1=0, t2=10*TORB, dt=0.01*TORB
  rk_inte1,elem,t1,t2,dt,/example,/plot
-----------------------------------------------------------------------------
\end{verbatim}

\small
\newpage

\black
\begin{verbatim}

;--------------------------------------------------
;1) Start by trying the /example values
;--------------------------------------------------
\end{verbatim}
\red
\begin{verbatim}
    rk_inte1,elem,t1,t2,dt,/example,/plot
\end{verbatim}
\black
\begin{verbatim}
; -plots an orbit integrated for 10 periods 
; -prints a table of orbital elements at different times
;  as well as dL/L, dE/E
; -and has also given values to variables: elem,t1,t2,dt

;  Check what are the example values of the orbital elements:
\end{verbatim}
\red
\begin{verbatim}
    print,elem            ;elem=[a,e,i,ome,w,tau]
    print,t1,t2,dt        ;in orbital periods
\end{verbatim}
\black
\begin{verbatim}
;Why does the orbit look like it does?

;--------------------------------------------------
;  same integrated orbit with small white squares=analytical orbit
\end{verbatim}
\red
\begin{verbatim}
    rk_inte1,elem,t1,t2,dt,/example,/plot,/cplot
\end{verbatim}
\black
\begin{verbatim}
;----------------------------------------------------------------------------
;2) Try now the same orbital elements (omit /example from the call)
;   with larger timesteps (The previous example had dt=0.01)
;   with oplot=1,2,3,... orbits from different calls are plotted on the same plot
\end{verbatim}
\red
\begin{verbatim}
     rk_inte1,elem,t1,t2,0.01,/plot,/cplot,/connect
     rk_inte1,elem,t1,t2,0.02,/plot,oplot=1,/connect
     rk_inte1,elem,t1,t2,0.03,/plot,oplot=3,/connect  ;starts to look bad?
     rk_inte1,elem,t1,t2,0.05,/plot,oplot=4,/connect  ;escape!
\end{verbatim}
\black
;remember: to get rid of windows use \red wide

\newpage
\black
\begin{verbatim}

;-------------------------------------------------------------------------------
;3) The integration thus gets progressively more inacurate with increasing time-step
;   Let's look how the error increases with time
;   t_out, e_out, l_out keywords return   dL/L and dE/E vs. time
;   output determines storing interval in steps
;   output=negative -> does not print to terminal

;3a) A short time interval (three orbits),
;    storing dL/L and dE/E at every step 
\end{verbatim}
\red
\begin{verbatim}
     !p.multi=[0,2,2]
     rk_inte1,elem,0.0,3.0,0.001,/plot,t_out=t_out,e_out=e_out,l_out=l_out,output=-1
     plot,t_out,e_out,psym=-4,xtitle='T/PER',ytitle='dE/E',title='dt=0.001 EKS=0.5'
     plot,t_out,l_out,psym=-4,xtitle='T/PER',ytitle='dL/L'
     !p.multi=0
\end{verbatim}
\black
\begin{verbatim}
;On what part of the orbit  does the error mainly occur?
;3b) Take a longer interval
\end{verbatim}
\red
\begin{verbatim}
     !p.multi=[0,2,2]
     rk_inte1,elem,0.,10,0.01,/plot,t_out=t_out,e_out=e_out,l_out=l_out,output=-20
     plot,t_out,e_out,psym=-4,xtitle='T/PER',ytitle='dE/E',title='dt=0.01,EKS=0.5'
     plot,t_out,l_out,psym=-4,xtitle='T/PER',ytitle='dL/L'
     !p.multi=0
\end{verbatim}
\black
\begin{verbatim}

;So, the overall error seems to increase linearly with time
;However, this is true only when the orbit does not change too much
;try larger dt=0.05   

\end{verbatim}
\red
\begin{verbatim}
     !p.multi=[0,2,2]
     rk_inte1,elem,0.,10,0.05,/plot,t_out=t_out,e_out=e_out,l_out=l_out,output=4,/connect
     plot,t_out,e_out,psym=-4,xtitle='T/PER',ytitle='dE/E',title='dt=0.05,EKS=0.5'
     plot,t_out,l_out,psym=-4,xtitle='T/PER',ytitle='dL/L'
     !p.multi=0

\end{verbatim}
\black
\begin{verbatim}

;why is there no accumulation of error in the end of this integration?

;-----------------------------------------------------------------------
;4) Let's plot the dL/L and dE/E (/per orbital period) versus timestep
;   These are returned by 'dl' and 'de' keywords   
;   Here is an example of how to do it compactly, without writing a procedure
;   So, just copy lines with mouse and move to IDL window!

\end{verbatim}
\red
\begin{verbatim}

   dt_tab=[.01,.005,.0025,.001]*1.d0
   dl_tab=dt_tab
   de_tab=dt_tab

   for i=0,n_elements(dt_tab)-1 do begin &  rk_inte1,elem,t1,t2,dt_tab(i),$
       plot=1,oplot=i,dl=dl,de=de & dl_tab(i)=dl & de_tab(i)=de & endfor

   nwin
   plot,dt_tab,abs(dl_tab),psym=-4,xtit='DT',ytit='dE/E, dL/L (/orbit)',title='RK4'
   oplot,dt_tab,abs(de_tab),psym=-6,col=2

\end{verbatim}
\black
\begin{verbatim}

;Error seems to increase very fast with time-step
;Try with log-log plot:

\end{verbatim}
\red
\begin{verbatim}

  nwin
  plot,dt_tab,abs(dl_tab),psym=-4,/xlog,/ylog,yr=[1d-15,1],xtit='DT',$
       ytit='dE/E, dL/L (/orbit)',title='RK4'
  oplot,dt_tab,abs(de_tab),psym=-6,col=2

\end{verbatim}
\black
\begin{verbatim}

;Errors seem to behave as dE/E = a * dt^k (which implies log(dE/E) = log(a)+k * log(dt)
;Try to determine k, by overplotting various lines:

\end{verbatim}
\red
\begin{verbatim}

  oplot,dt_tab,dt_tab^2,col=3,lines=2
  oplot,dt_tab,dt_tab^3,col=5,lines=2
  oplot,dt_tab,dt_tab^4,col=6,lines=2
  oplot,dt_tab,dt_tab^5,col=7,lines=2
  oplot,dt_tab,dt_tab^6,col=8,lines=2

\end{verbatim}
\black
\begin{verbatim}

;Which curve seems to have the right slope?

\end{verbatim}
\red
\begin{verbatim}

  oplot,dt_tab,de_tab(0)*(dt_tab/dt_tab(0))^5,psym=-1,thick=3,sym=1,lines=2  
  ; almost perfect?

\end{verbatim}
\black
\begin{verbatim}
;How does the result agree with the RK4 being a fourth-order method?


;-----------------------------------
;5) So, errors in dL/L and dE/E depends on timestep
;   How does it depend on other things, like a and eks?


;5a) try different semimajor-axis:

\end{verbatim}
\red
\begin{verbatim}
  elem1=[1.0,  0.5,   0.,  0.,  0.,  0.]		;elem=[a,e,i,ome,w,tau]
  elem2=[2.0,  0.5,   0.,  0.,  0.,  0.]		;so a is changed from a=1 to a=2

  dt=0.01
  t1=0.
  t2=10.

  rk_inte1,elem1,t1,t2,dt,dl=dl1,de=de1,/plot,wid=4
  rk_inte1,elem2,t1,t2,dt,dl=dl2,de=de2,/plot,oplot=1           
  ;added /plot to check the change in orbit

\end{verbatim}
\black
\begin{verbatim}
; check the changes in E and L
; Is there any difference?
  
\end{verbatim}
\red
\begin{verbatim}
  print,'error dE/E', de1,de2 
  print,'error dL/L', dl1,dl2


\end{verbatim}
\black
\begin{verbatim}

;5b) try different eccentricities:

\end{verbatim}
\red
\begin{verbatim}

  elem1=[1.0,  0.5,   0.,  0.,  0.,  0.]		;elem=[a,e,i,ome,w,tau]
  elem2=[1.0,  0.05,   0.,  0.,  0.,  0.]		;so eks is reduced form eks=0.5 to eks=0.05

  dt=0.01
  t1=0.
  t2=10.

  rk_inte1,elem1,t1,t2,dt,dl=dl1,de=de1,/plot,wid=4
  rk_inte1,elem2,t1,t2,dt,dl=dl2,de=de2,/plot,oplot=1 

\end{verbatim}
\black
\begin{verbatim}
; check the changes in E and L

\end{verbatim}
\red
\begin{verbatim}

  print,de1,de2
  print,dl1,dl2

\end{verbatim}
\black
\begin{verbatim}
; So, how does the eccentricity affect?

;5c) Let's try for a range of different values
;    choose dt=0.001 to avoid too large changes with eks=0.9! 

\end{verbatim}
\red
\begin{verbatim}

   eks_tab=[.001,.01,.1,.25,.5,.75,.9,.95]*1.d0
   dl_tab=eks_tab
   de_tab=eks_tab
   t1=0.
   t2=10.
   dt=0.001
   output=1000   ; not to print so much
  
   for i=0,n_elements(eks_tab)-1 do begin &  rk_inte1,[1.,eks_tab(i),0.,0.,0.,0.],$
       t1,t2,dt,plot=1,oplot=i,wid=5,output=output,dl=dl,de=de $
       & dl_tab(i)=dl & de_tab(i)=de & endfor


   nwin
   plot,eks_tab,abs(dl_tab),/xlog,/ylog,psym=-4,xtitle='eccentricty',$
        ytitle='error',yr=[1d-16,1.]
   oplot,eks_tab,abs(de_tab),psym=-6,col=2

\end{verbatim}
\black
\begin{verbatim}

;so things get really bad when eccentricity is increased
;try to look it this way

\end{verbatim}
\red
\begin{verbatim}

   nwin
   plot,1.-eks_tab,abs(dl_tab),/xlog,/ylog,psym=-4,xtitle='1-eccentricty',$
        ytitle='error',yr=[1d-16,10.]
   oplot,1.-eks_tab,abs(de_tab),psym=-6,col=2

\end{verbatim}
\black
\begin{verbatim}
;approximative fit?

\end{verbatim}
\red
\begin{verbatim}

   apu=lindgen(100)*.01+.01
   oplot,apu,1d-12/apu^8,lines=2,col=3

\end{verbatim}
\black
\begin{verbatim}

;Remembering the result in 3a) how would you intepret this stong eccentricity dependence?

;just plotting the time evolution with EKS=0.9 and EKS=0.5

\end{verbatim}
\red
\begin{verbatim}

     elem1=[1.,0.9,0.,0.,0.,0.]
     rk_inte1,elem1,0.0,3.0,0.001,/plot,$
              t_out=t_out1,e_out=e_out1,l_out=l_out1,output=-1

     elem2=[1.,0.5,0.,0.,0.,0.]
     rk_inte1,elem2,0.0,3.0,0.001,/plot,oplot=1,$
              t_out=t_out2,e_out=e_out2,l_out=l_out2,output=-1

     !p.multi=[0,2,1]
     nwin
     plot,t_out1,e_out1,xtit='T/PER',ytit='dE/E',$
          title='dt=0.001 EKS=0.9 (red) 0.1 (GREEN)',col=2,/ylog,yr=[1d-12,1]
     oplot,t_out2,e_out2,col=3
     plot,t_out1,-l_out1,xtit='T/PER',ytit='dL/L',col=2,/ylog,yr=[1d-12,1]
     oplot,t_out2,-l_out2,col=3
     !p.multi=0

\end{verbatim}
\black
\begin{verbatim}
;-----------------------------------------------------------------------------------
;6 Let's now have a look at errors in Taylor-series integration
;  keyword taylor defines the degree (1 or 2)

     
;6a) TAYLOR I

\end{verbatim}
\red
\begin{verbatim}

     elem=[1.,0.5,0.,0.,0.,0.]

     dt=0.01
     rk_inte1,elem,0.0,1.0,dt,/plot,output=-1,/cplot,taylor=1,$
        title='Taylor I: dt=0.01, 0.001, 0.0001'    
     ;terrible

     dt=0.001
     rk_inte1,elem,0.0,1.0,dt,/plot,output=-1,taylor=1,oplot=1 
     ; bad

     dt=0.0001
     rk_inte1,elem,0.0,1.0,dt,/plot,output=-1,taylor=1,oplot=2 
     ; still bad, has not completed one orbit!

\end{verbatim}
\black
\begin{verbatim}
;6b) TAYLOR II

\end{verbatim}
\red
\begin{verbatim}
     elem=[1.,0.5,0.,0.,0.,0.]

     dt=0.01
     rk_inte1,elem,0.0,1.0,dt,/plot,output=-1,/cplot,taylor=2,$
         title='Taylor 2: dt=0.01, 0.001, 0.0001'    
     ; bad

     dt=0.001
     rk_inte1,elem,0.0,1.0,dt,/plot,output=-1,taylor=2,oplot=1 
     ; better

     dt=0.0001
     rk_inte1,elem,0.0,1.0,dt,/plot,output=-1,taylor=2,oplot=2 
     ; almost acceptable


\end{verbatim}
\black
\begin{verbatim}
;-------------------------------------------
;6c) Let's check the error vs. dt dependence, as we did for RK4

;with taylor II
;if this seems to take forever, remove the shortest time-step

\end{verbatim}
\red
\begin{verbatim}

   elem=[1.,0.5,0.,0.,0.,0.]
   dt_tab=[.01,.001,.0001,.00001]*1.d0
   dl_tab=dt_tab
   de_tab=dt_tab
   t1=0.d0
   t2=2.d0

   for i=0,n_elements(dt_tab)-1 do begin &  rk_inte1,elem,t1,t2,dt_tab(i),$
       plot=0,oplot=i,dl=dl,de=de,tay=2 & dl_tab(i)=dl & de_tab(i)=de & endfor

  nwin
  plot,dt_tab,abs(dl_tab),psym=-4,/xlog,/ylog,yr=[1d-12,1],xtit='DT',$
       ytit='dE/E, dL/L (/orbit)',title='TAylor II, EKS='+string(elem(1))
  oplot,dt_tab,abs(de_tab),psym=-6,col=2


\end{verbatim}
\black
\begin{verbatim}
;again, check the exponent of DE/E vs dt^k dependence

\end{verbatim}
\red
\begin{verbatim}
  oplot,dt_tab,1d3*dt_tab^2,lines=2
  oplot,dt_tab,1d3*dt_tab^3,lines=2

\end{verbatim}
\black
\begin{verbatim}
;Error proportional to dt^-3 ?  

;------------------------------------
;Repeat the same with eks=0.05
;let's plot them on top of the previous error-curves

\end{verbatim}
\black



\end{document}
