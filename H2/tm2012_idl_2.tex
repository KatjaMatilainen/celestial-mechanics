\documentclass[a4paper,12pt]{article}
%\documentclass[a4,landscape]{seminar}
%\slidesmag{1}
%\documentstyle[a4paper]{letter}
%\documentstyle[fullpage]{letter}
\usepackage{graphicx} 
\usepackage{amsmath} 

\usepackage{color}
%\usepackage{pstcol} % To use the standard `color' package with Seminar
%\usepackage{semcolor}
%\usepackage{slidesec}
%\input{seminar.bug}
%\input{seminar.bg2} % See the Seminar bugs list
%\usepackage{pst-grad}

\definecolor{LemonChiffon}{rgb}{1.,0.98,0.8}
\definecolor{Gold}{rgb}{1.,0.84,0.}
\definecolor{darkblue}{rgb}{0.25,0.,0.95}
\definecolor{lightblue}{rgb}{0.4,0.65,0.95}
%\slideframe[\psset{fillstyle=gradient,gradmidpoint=0.0,
%                   gradbegin=lightblue,gradend=darkblue}]{scplain}

\def\pp{\mathaccent "7F }
\def\p{\mathaccent 95 }


\def\sini {\sin i}
\def\cosi {\cos i}

\def\sino {\sin \Omega}
\def\coso {\cos \Omega}

\def\sinw {\sin \omega}
\def\cosw {\cos \omega}



\begin{document}
\font\norm=cmr12
\font\hugeb=cmb22
\font\isob=cmb18
\font\iso=cmr18
\font\med=cmr14
\font\medb=cmb14
\baselineskip 0.6cm
\parskip\medskipamount
\parskip 0.35cm
\parindent 5mm

%\boldmath

\norm
\hsize 16cm
\vsize 30cm
\newcommand{\bul}{$\bullet \ \ $}
\newcommand{\buu}{\hskip 0.5cm}
\newcommand{\arrow}{$\Rightarrow \ \ $}
\newcommand{\barrow}{\hskip 0.5cm $\Rightarrow \ \ $}
\newcommand{\larrow}{\hskip 0.5cm $\Leftarrow \ \ $}
\newcommand{\hlin}{\vskip 0.5cm}
\newcommand{\hpar}{\vskip 1.0cm}
\newcommand{\nin}{\noindent}
\def\pii{\tilde{\omega}_\circ}

\newcommand{\VV}{$\vec V$ \hskip 0.1cm}
\newcommand{\RR}{$\vec R$ \hskip 0.1cm}

\newcommand{\VVE}{\vec V}
\newcommand{\RRE}{\vec R}

%\pagestyle{empty}
{\centerline{}




{\norm
\vskip 0cm
{{\centerline { {\isob CELESTIAL MECHANICS (Fall 2012): }}}}
\vskip 0.2cm
{{\centerline { {\isob COMPUTER EXERCISES II}}}}

{{\centerline { { (Heikki Salo 2.11.2012)}}}}

\vskip 2cm

{\medb 5. Calculation of R and V from orbital elements:}

{\medb 6. Calculation of orbital elements from R and V}

{\medb 7. Playing with orbits}



\vskip 1cm

Example routines for the solution of the exercises can be copied from

\begin{verbatim}
/wrk/hsalo/TM2012_idl.dir/TM2012_DEMO2.dir
\end{verbatim}



\newpage

%-------------------------------------------------------------------------
\newpage
{\isob 5. Calculation of \RR and \VV from orbital elements} \ (elem\_to\_rv.pro)

The previous examples have dealt with elliptic motion in a coordinate
system aligned with the orbital plane, with the x-axis pointing to
the pericenter. Let's denote the Cartesian coordinates in this system by
$\tilde x$ and $\tilde y$, and the velocities by $\tilde v_x$ and $\tilde
v_y$. The calculation of these coordinates for a given time $t$ required
just $a, \epsilon$ and $\tau$: the shape and size of the orbit is
given by $\epsilon$ and $a$, while $\tau$ gave the instant of
pericenter passage. 

Here we calculate the position \RR and velocity \VV vectors with
respect to a reference coordinate system (e.g. Earth's orbital plane,
with the x-axis pointing to the vernal equinox), starting from the
full set of orbital elements $(a, \epsilon, i, \Omega, \omega, \tau)$. The
orientation of the planet's orbital plane is defined with $\Omega$ and
$i$, the longitude of node with respect to a fixed direction along the
xy-reference plane, and the inclination of the orbit, respectively.
The orientation of the pericenter in the orbital plane is defined with
$\omega$, the argument of pericenter, giving the angle of the
pericenter from the nodal line (along the orbital plane).


{\bf ELLIPTIC ORBIT: }
The following steps are required (steps 1-3 were already
needed in previous exercises, but are collected here to list a complete set of formulas):

{\bf 1)} Calculate the period, and the mean anomaly $M$
\begin{eqnarray}
P &=& 2\pi \sqrt{a^3/\mu} \\ \nonumber
M &=& \frac{2\pi}{P}(t-\tau)
\label{period}
\end{eqnarray}

{\bf 2)} Solve the eccentric anomaly $E$ from Kepler's equation
\begin{equation}
M = E - \epsilon \sin(E)
\end{equation}

{\bf 3)} Calculate the orbital plane coordinates $\tilde x, \tilde y$ and $\tilde v_x, \tilde v_y$:
\begin{eqnarray}
{\tilde x} &=& a (\cos E - \epsilon) \\ \nonumber
{\tilde y} &=& b  \sin E             \\ \nonumber 
{\tilde v_x} &=& -a \sin E  ~~ \sqrt{\mu/a^3} \ (1-\epsilon \cos E)^{-1}    \\ \nonumber
{\tilde v_y} &=&  ~~ b \cos E ~~ \sqrt{\mu/a^3} \  (1-\epsilon \cos E)^{-1}      \\ \nonumber
\end{eqnarray}

\newpage
{\bf 4)} All the extra which needs to be done is to rotate the components of
position and velocity to the reference system (of course $\tilde z=0$ and $\tilde v_z=0$ along the orbit):
\begin{eqnarray}
\vec R &=& \ \tilde x \left( \frac{\vec A}{a}\right) + \ \tilde y \left( \frac{\vec B}{b}\right) + \ \tilde z \vec N \\ \nonumber
\vec V &=& \tilde v_x\left( \frac{\vec A}{a}\right)  + \tilde v_y \left( \frac{\vec B}{b}\right) + \ \tilde v_z \vec N
\end{eqnarray}
\noindent where the directions of $\vec A$, $\vec B$, and $N= \left( \frac{\vec A}{a}\right) \times \left( \frac{\vec B}{b}\right)$ in the reference system
 (obtained by a rotation of the orbital plane-  coordinate system by $-\omega$ with respect to the z-axis, $-i$ with respect to the new x-axis, and by $-\Omega$ around the new z-axis) are given by:

\begin{equation}
\frac{\vec A}{a} = \begin{pmatrix} \cosw \coso - \sinw \sino \cosi \\ \cosw \sino + \sinw \coso \cosi \\ \sinw \sini \end{pmatrix}
\end{equation}

\begin{equation}
\frac{\vec B}{b} = \begin{pmatrix} -\sinw \coso - \cosw \sino \cosi \\ -\sinw \sino + \cosw \coso \cosi \\ \cosw \sini \end{pmatrix}
\end{equation}

\begin{equation}
\vec N = \begin{pmatrix} ~~\sino \sini \\ -\coso \sini \\ \cosi \end{pmatrix}
\label{eq_n}
\end{equation}

\vskip 1cm
{\bf HYPERBOLIC ORBIT:}
The treatment of hyperbolic orbit can be incorporated to the same
procedure: the only changes are in steps 2 and 3. In the case of hyperbolic
orbit $(\epsilon > 1)$, the Kepler's equation is replaced by
\begin{equation}
 M = \epsilon \sinh E  - E
\label{eq_kepler_hyp} 
\end{equation}
\noindent which can be solved with similar methods as in the elliptic case
(see {\bf hkepler.pro}). 

\newpage
\noindent Instead of step 3 we have:
\begin{eqnarray}
{\tilde x} &=& a (\epsilon - \cosh E) \\ \nonumber
{\tilde y} &=& b  \sinh E             \\ \nonumber 
{\tilde v_x} &=& -a \sinh E ~~ \sqrt{\mu/a^3} \ (\epsilon \cosh E -1)^{-1}    \\ \nonumber
{\tilde v_y} &=&  ~~ b \cosh E  ~~ \sqrt{\mu/a^3} \  (\epsilon \cosh E -1)^{-1},      \\ \nonumber
\end{eqnarray}
\noindent obtained from the vectorial representation of hyperbolic orbit:
\begin{equation}
\vec R = \vec A (\epsilon - \cosh E) + \vec B \sinh E
\label{eq_vec_r_hyp}
\end{equation}
\begin{equation}
\vec V =\p {\vec R} = (- \vec A \sinh E + \vec B \cosh E) \p E,
\label{eq_vec_v_hyp}
\end{equation}

\noindent where $\p E$ is get from differentiating Eq. 
\ref{eq_kepler_hyp}, giving
\begin{equation}
\p E = \frac{\sqrt{\mu/a^3}}{(\epsilon \cosh E -1)}
\label{eq_ep_hyp}
\end{equation}
\noindent Also remember that for hyperbolic orbit $b=a \sqrt{\epsilon^2-1}$

\vskip 1cm
{\bf WHAT TO DO: }
Make an IDL-procedure performing the above steps, taking as input
variables the set of orbital elements
$(a,\epsilon,i,\Omega,\omega,\tau)$ and time $t$.  The units of time
and distances can all be specified by the variable $\mu$. For example,
if we study the motion in the Solar system, then we know that the
orbital period equals one year at the distance of Earth's orbit.
According to Eq. \ref{period} we must thus use $\mu=4 \pi^2$ if time
is measured in years and distances in Astronomical units. Other
choices are also possible: for example if we choose $\mu=1$, then
according to Eq. \ref{period}, the orbital period is $2 \pi$ at the
unit distance.  This is a very natural choice, as then the circular
orbit velocity and angular velocity also equal unity at the unit distance.


\vskip 1cm
\newpage
{\isob 6. Calculation of orbital elements from $\vec R$ and $\vec V$} \ (rv\_to\_elem.pro)

Here we look at the inverse problem, i.e. the calculation of orbital
elements from the position and velocity vectors, specified in the reference
system (e.g. Earth's orbital plane). 

{\bf ELLIPTIC ORBIT: }

{\bf 1)} Calculate the unit vector in the direction perpendicular to the
orbital plane, from the vector $\vec K=\vec R \times \vec V$:
\begin{equation}
 \vec N = \frac{\vec R \times \vec V}{|\vec R \times \vec V|},
\end{equation}
\noindent from which we get (from Eq. \ref{eq_n})
\begin{eqnarray}
\label{eq_ome_i}
\Omega &=& {\rm atan} (N_x,-N_y) \\ \nonumber
 i &=& {\rm acos} (N_z)
\end{eqnarray}

{\bf 2)} Calculate the eccentricity vector $\vec e$,
\begin{equation}
 \vec e = -\frac{1}{\mu} \vec K \times \vec V - \vec R/r. 
\end{equation}
\noindent Then rotate $\vec e$  \ from the reference system to the
orbital plane-coordinates (involves rotation by $\Omega$ around the z-axis,
followed by rotation by $i$ around the new x-axis):
\begin{eqnarray}
\tilde e_x &=& ~~ \coso \ e_x + \sino \ e_y \\ \nonumber
\tilde e_y &=& - \sino \cosi \ e_x + \coso \cosi  \ e_y + \sini \ e_z,
\end{eqnarray}
\noindent and finally
\begin{eqnarray}
\omega &=& {\rm atan} (\tilde e_y, \tilde e_x) \\ \nonumber
 \epsilon &=& \sqrt{{\tilde e_x}^2+{\tilde e_y}^2}
\end{eqnarray}

{\bf 3)} Semi-major axis $a$ is obtained from the energy equation, $h=\frac{1}{2}v^2-\mu/r = -\frac{\mu}{2a}$ for elliptic orbit, giving
\begin{equation}
 a = -\frac{1}{v^2/\mu - 2/r}.
\label{energy}
\end{equation}

{\bf 4)} Time of pericenter passage is obtained by calculating
first the eccentric anomaly and then using Kepler's equation.
Since $r=a(1-\epsilon \cos E)$ we obtain  
\begin{equation}
\cos E = \frac{(1 - r/a)}{\epsilon}
\end{equation}
\noindent This alone does not determine $E$ uniquely. The correct sign of $E$ is determined by the sign of $\vec R \cdot \vec V$ (for $0^\circ < E < 180^\circ$ we have $\vec R \cdot \vec V>0$)
\begin{equation}
E = {\rm acos}(\cos E) ~~ {\rm sign}(\vec R \cdot \vec V)
\end{equation}
\noindent Then calculate $M$ and finally $\tau$,
\begin{eqnarray}
M &=& E - \epsilon \sin E \\ \nonumber
 \tau &=& t - M \sqrt{a^3/\mu} 
\end{eqnarray}

{\bf HYPERBOLIC ORBIT: } The treatment of hyperbolic orbit (the type
of orbit is identified in the step 2 above if $\epsilon>1$) is again
very similar to elliptic orbit.  Only the steps 3 and 4 are
different. Since in the case of hyperbolic orbit the total energy is
positive, $h=\frac{\mu}{2a}$, the sign in Eq. \ref{energy} needs to be
reversed. Thus both elliptic and hyperbolic case can be treated if we
use
\begin{equation}
 a = \frac{1}{|v^2/\mu - 2/r|}
\end{equation}
The step 4) needs to replaced by
\begin{eqnarray}
\cosh E &=& \frac{(1 + r/a)}{\epsilon} \\ \nonumber
E &=& {\rm acosh}(cosh E)  ~~  {\rm sign}(\vec R \cdot \vec V) \\ \nonumber
M &=& \epsilon \sinh E  -E \\ \nonumber
 \tau &=& t - M \sqrt{a^3/\mu} 
\end{eqnarray}


{\bf WHAT TO DO: } Make an IDL-procedure performing the above steps,
taking as input variables the vectors $\vec R$ and $\vec V$ and time,
and returning the set of orbital elements
$(a,\epsilon,i,\Omega,\omega,\tau)$.  The units of time and distances
are specified by the variable $\mu$, as discussed earlier in the case
of obtaining $\vec R$ and $\vec V$ from orbital elements.

\newpage
{\isob 7. Playing with orbits}


Copy the example IDL procedures for the solution of exercises 5 and 6
to your own directory:

{\tt cp /wrk/hsalo/TM2012\_DEMO2.dir/* .}

\noindent These routines include: (give the name without arguments to obtain an info message)


{\parskip 0.05cm


\bul {\bf elem\_to\_rv.pro} -- calculate \RR and \VV from orbital elements (Exercise 5)

\bul {\bf rv\_to\_elem.pro} -- calculate orbital elements from \RR and \VV (Exercise 6)

\bul {\bf elem\_to\_example.pro} -- Check the above two routines:

\buu  first calculate \RR and \VV for a given time from orbital elements, 

\buu then transform \RR and \VV back to orbital elements (hopefully the same!)


\bul {\bf kepler.pro} -- solves Kepler's equation \ \ ($M$, $\epsilon$ \arrow $E$)

\bul {\bf kepler\_array.pro} -- solves Kepler's equation for an array of M values

\bul {\bf hkepler.pro} -- as kepler.pro, except for a hyperbolic orbit ($\epsilon>1$
}

\noindent + auxillary mathematical routines missing from basic IDL-routines

{\parskip 0.15cm

\bul {\bf cross\_product.pro} -- calculate vector product of two 3-element vectors

\bul {\bf dot\_product.pro} -- calculate scalar product of two 3-element vectors

\bul {\bf acosh.pro} -- function returning inverse hyperbolic cosine
}


Use the above routines in the next exercises (or better, write your own equivalent routines!)



\vskip 0.5cm

{\medb a) Procedure for calculating + plotting an orbit} 


Make a subroutine-type procedure (call it for example {\tt elem\_orbit\_oma.pro})
for calculating/plotting an orbit with given
orbital elements  ({\tt elem=$[a, \epsilon, i, \Omega, \omega, \tau]$}), over an interval of time $t_1, t_2$.

Hint: make repeated calls to {\bf elem\_to\_rv}, after checking what type
of input/output it uses. Pay attention to the units you use.

You may also have a look at the example procedure {\bf elem\_orbit.pro}




\newpage


{\medb b) Satellite launched from Earth orbit} 

Make a main-program type procedure, where a velocity increment is
given  to a satellite at Earth orbit:

{\parskip 0.1 cm
- in the tangential direction

- in the radial direction

- in the vertical direction
}

Calculate the orbital elements of the satellite. Assume that Earth
orbit is circular with speed $30$ km/sec, and use different velocity
increments in the range 0-30 km/sec.
Use the procedure {\bf elem\_orbit.pro} (or the equivalent procedure you have written)

For examples of solutions, see

{\parskip 0.1 cm
\bul {\bf elem\_orbit\_demo\_tangential.pro}

\bul {\bf elem\_orbit\_demo\_radial.pro}

\bul {\bf elem\_orbit\_demo\_vertical.pro} 

\bul {\bf elem\_orbit\_demo\_vertical.pro} -- illustrates the effect of vertical
velocity increment using IDL's interactive xplot3d-plotting procedure
}


Check also other example routines:

\bul {\bf elem\_rv\_demo1.pro} -- Solves the Exercise II.1 (satellite
launced with a velocity vector making different angles with the
tangential direction}

\bul {\bf elem\_rv\_demo2.pro} -- shows how a 'swarm of satellites'
launched from Earth disperses with time.




%-------------------------------------------------------------------------



\end{document}









